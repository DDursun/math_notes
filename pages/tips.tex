\twocolumn[
\begin{center}
\textbf{\Large General shortcuts}
\end{center}
\vspace{4pt}]


It is often the case that complex problems are transferred into a different domain (e.g., via Laplace transforms) or reformulated as systems of ODEs, and complex algebra is the price we pay to solve such problems. This section provides tips and tricks to help solve some of the common problems encountered.

\hrulefill


\textbf{Partial Fraction Decomposition}

Being able to x the equation into fractions is a useful tool in many situations. To achieve this, set up the expected form, multiply both sides by the common denominator, then either:
(1) Substitute convenient values (roots ofthe  denominator) to find the coefficients, or
(2) Expand and match coefficients of like powers.
Common patterns:

$\dfrac{1}{(s+a)(s+b)} = \dfrac{A}{s+a} + \dfrac{B}{s+b}$ 

$\dfrac{1}{s(s+a)} = \dfrac{A}{s} + \dfrac{B}{s+a}$

$\dfrac{1}{s^2(s+a)} = \dfrac{A}{s} + \dfrac{B}{s^2} + \dfrac{C}{s+a}$

$\dfrac{1}{(s+a)^n} = \dfrac{A_1}{s+a} + \dfrac{A_2}{(s+a)^2} + \cdots + \dfrac{A_n}{(s+a)^n}$

$\dfrac{1}{(s+a)^n} = \dfrac{A_1}{s+a} + \dfrac{A_2}{(s+a)^2} + \cdots + \dfrac{A_n}{(s+a)^n}$

$\dfrac{1}{(s+a)^2(s+b)} = \dfrac{A}{s+a} + \dfrac{B}{(s+a)^2} + \dfrac{C}{s+b}$

$\dfrac{1}{(s+a)(s^2+b^2)} = \dfrac{A}{s+a} + \dfrac{Bs+C}{s^2+b^2}$




\hrulefill

\textbf{Polynomial Long Division}

Once one root of a cubic is found, polynomial division reduces it to a quadratic, whose remaining roots follow easily. 
If the constant term is not large, try integer roots in $[-2,2]$ first. Then,


Divide leading terms $\to$ multiply divisor by result $\to$ subtract $\to$ bring down $\to$ repeat.

Ex: $(x^3 - x^2 - x - 2) \div (x - 2)$
\[
\arraycolsep=1pt
\begin{array}{r|l}
x^2 + x + 1 & \\
\hline
x - 2 \;\big) & x^3 - x^2 - x - 2 \\
& x^3 - 2x^2 \downarrow \\
\cline{2-2}
& x^2 - x \\
& x^2 - 2x \downarrow \\
\cline{2-2}
& x - 2 \\
& x - 2 \\
\cline{2-2}
& 0
\end{array}
\]
Result: $x^2 + x + 1$


\textbf{17. Useful Trigonometric Equalities}

$\sinh x = \frac{e^x - e^{-x}}{2}$, \quad $\cosh x = \frac{e^x + e^{-x}}{2}$

$\cosh^2 x - \sinh^2 x = 1$, \quad $\cosh^2 x + \sinh^2 x = \cosh(2x)$

$\sinh(A \pm B) = \sinh A \cosh B \pm \cosh A \sinh B$

$\cosh(A \pm B) = \cosh A \cosh B \pm \sinh A \sinh B$

$\frac{d}{dx}\sinh x = \cosh x$, \quad $\frac{d}{dx}\cosh x = \sinh x$



\textbf{18. Reduction Formulas}

$\displaystyle\int \sin^n\theta\,d\theta = -\frac{\sin^{n-1}\theta\cos\theta}{n} + \frac{n-1}{n}\int\sin^{n-2}\theta\,d\theta$

$\displaystyle\int \cos^n\theta\,d\theta = \frac{\cos^{n-1}\theta\sin\theta}{n} + \frac{n-1}{n}\int\cos^{n-2}\theta\,d\theta$

$\displaystyle\int \tan^n\theta\,d\theta = \frac{\tan^{n-1}\theta}{n-1} - \int\tan^{n-2}\theta\,d\theta$

$\displaystyle\int \sec^n\theta\,d\theta = \frac{\sec^{n-2}\theta\tan\theta}{n-1} + \frac{n-2}{n-1}\int\sec^{n-2}\theta\,d\theta$





\textbf{Polar Coordinates}


\textbf{6. Chain Rule}

$\frac{dz}{dt} = \frac{\partial f}{\partial x}\frac{dx}{dt} + \frac{\partial f}{\partial y}\frac{dy}{dt}$, \quad $\frac{\partial z}{\partial s} = \frac{\partial f}{\partial x}\frac{\partial x}{\partial s} + \frac{\partial f}{\partial y}\frac{\partial y}{\partial s}$


\textbf{Euler's Formula \& Identity}

\[
e^{i\theta} = \cos\theta + i\sin\theta
\]
\[e^{i\pi} + 1 = 0\]
