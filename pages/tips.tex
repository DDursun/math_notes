\twocolumn[
\begin{center}
\textbf{\Large Useful Identities \& Tricks}
\end{center}
\vspace{4pt}]


It is often the case that complex problems are transferred into a different domain (e.g., via Laplace transforms) or reformulated as systems of ODEs, and complex algebra is the price we pay to solve such problems. This section provides tips and tricks to help solve some of the common problems encountered.

\hrulefill


\textbf{1. Partial Fraction Decomposition}

Being able to split the equation into fractions is a useful tool. To achieve this, set up the expected form, multiply both sides by the common denominator, then either:
(1) Substitute convenient values (roots of the  denominator) to find the coefficients, or
(2) Expand and match coefficients of like powers.
Common patterns:

$\dfrac{1}{(s+a)(s+b)} = \dfrac{A}{s+a} + \dfrac{B}{s+b}$ 

$\dfrac{1}{s^2(s+a)} = \dfrac{A}{s} + \dfrac{B}{s^2} + \dfrac{C}{s+a}$

$\dfrac{1}{(s+a)^n} = \dfrac{A_1}{s+a} + \dfrac{A_2}{(s+a)^2} + \cdots + \dfrac{A_n}{(s+a)^n}$


$\dfrac{1}{(s+a)^2(s+b)} = \dfrac{A}{s+a} + \dfrac{B}{(s+a)^2} + \dfrac{C}{s+b}$

$\dfrac{1}{(s+a)(s^2+b^2)} = \dfrac{A}{s+a} + \dfrac{Bs+C}{s^2+b^2}$




\hrulefill

\textbf{2. Polynomial Long Division}

Once one root of a cubic is found, polynomial division reduces it to a quadratic, whose remaining roots follow easily. 
If the constant term is not large, try integer roots in $[-2,2]$ first. Then,


Divide leading terms $\to$ multiply divisor by result $\to$ subtract $\to$ bring down $\to$ repeat.

Ex: $(x^3 - x^2 - x - 2) \div (x - 2)$
\[
\arraycolsep=1pt
\begin{array}{r|l}
x^2 + x + 1 & \\
\hline
x - 2 \;\big) & x^3 - x^2 - x - 2 \\
& x^3 - 2x^2 \downarrow \\
\cline{2-2}
& x^2 - x \\
& x^2 - 2x \downarrow \\
\cline{2-2}
& x - 2 \\
& x - 2 \\
\cline{2-2}
& 0
\end{array}\]
Result: $x^2 + x + 1$

\textbf{3. Equalizing indices in sum operators}

Equalizing indices (or ``shifting indices'') in sum operators ($\sum$) involves adjusting the lower/upper bounds and the general term formula simultaneously to make summation limits match, often for combining terms.


 \textbf{Shifting the Index:} To shift the starting index from $m$ to $m+r$, replace $n$ with $n+r$ in the expression and subtract $r$ from the upper bound to keep the number of terms consistent:
    \[
    \sum_{n=m}^{p} a_n = \sum_{n=m+r}^{p+r} a_{n-r}
    \]

The idea is simple: if a backward shift is needed (starting the sum from $n-1$ instead of $n$), replace the lower limit $n \to n-1$, substitute $n \to n+1$ in the internal expression to compensate.

\textbf{Example:} To shift $\sum_{n=1}^{4} n^2$ to start at $n=0$, it becomes $\sum_{n=0}^{3} (n+1)^2$.


\textbf{4. Useful Trigonometric Equalities}

$\sinh x = \frac{e^x - e^{-x}}{2}$, \quad $\cosh x = \frac{e^x + e^{-x}}{2}$

$\cosh^2 x - \sinh^2 x = 1$, \quad $\cosh^2 x + \sinh^2 x = \cosh(2x)$

$\sinh(A \pm B) = \sinh A \cosh B \pm \cosh A \sinh B$

$\cosh(A \pm B) = \cosh A \cosh B \pm \sinh A \sinh B$

$\frac{d}{dx}\sinh x = \cosh x$, \quad $\frac{d}{dx}\cosh x = \sinh x$



\textbf{5. Reduction Formulas}

$\displaystyle\int \sin^n\theta\,d\theta = -\frac{\sin^{n-1}\theta\cos\theta}{n} + \frac{n-1}{n}\int\sin^{n-2}\theta\,d\theta$

$\displaystyle\int \cos^n\theta\,d\theta = \frac{\cos^{n-1}\theta\sin\theta}{n} + \frac{n-1}{n}\int\cos^{n-2}\theta\,d\theta$

$\displaystyle\int \tan^n\theta\,d\theta = \frac{\tan^{n-1}\theta}{n-1} - \int\tan^{n-2}\theta\,d\theta$

$\displaystyle\int \sec^n\theta\,d\theta = \frac{\sec^{n-2}\theta\tan\theta}{n-1} + \frac{n-2}{n-1}\int\sec^{n-2}\theta\,d\theta$

\textbf{6. Euler's Formula \& Identity}
\[
e^{i\theta} = \cos\theta + i\sin\theta, \quad e^{i\pi} + 1 = 0
\]



\textbf{7. Chain Rule}
This rule applies to composite functions, where one function is nested inside another. The idea is similar to connected gears: to find the total rate of change, you multiply the rate of change of the outer function by the rate of change of the inner function.

To differentiate $y = f(g(x))$, let $u = g(x)$. Then $y = f(u)$ and:
\[
\frac{dy}{dx} = \frac{dy}{du} \times \frac{du}{dx}
\]

For multivariable functions, the chain rule involves summing contributions from each intermediate variable:
\[
\frac{dz}{dt} = \frac{\partial f}{\partial x}\frac{dx}{dt} + \frac{\partial f}{\partial y}\frac{dy}{dt}, \quad \frac{\partial z}{\partial s} = \frac{\partial f}{\partial x}\frac{\partial x}{\partial s} + \frac{\partial f}{\partial y}\frac{\partial y}{\partial s}
\]

\textbf{8. Polar Coordinates}
 A point is defined by distance $r$ from the origin and angle $\theta$ from the positive $x$-axis. Useful for circular/rotational problems and simplifies integrals with radial symmetry.

 \begin{wrapfigure}{r}{0.23\textwidth}
    \centering
    \vspace{-15pt}
    \includegraphics[width=0.24\textwidth]{figures/polar.png}
    \caption{\scriptsize Polar  coordinate system}
    \vspace{-5pt}
\end{wrapfigure}

\textbf{Conversions:}
\begin{align*}
x &= r\cos\theta, \quad y = r\sin\theta \\
r &= \sqrt{x^2 + y^2}, \quad \theta = \arctan\frac{y}{x}
\end{align*}

\textbf{Line and Area elements:} 

$ds = \sqrt{dr^2 + r^2 d\theta^2}$

$dA = r \, dr \, d\theta$