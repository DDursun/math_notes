\twocolumn[
\begin{center}
\textbf{\Large Laplace Transforms}
\end{center}
\vspace{4pt}]
%----------------------------------------------
\textbf{Definition of a Laplace Transform}

The \textbf{Laplace transform} of a function $f(t)$ is defined as
\[
\mathcal{L}\{f(t)\} = F(s) = \int_{0}^{\infty} e^{-st} f(t)\,dt.
\]

In simple terms, the Laplace transform maps a function from the time domain $t$ to the complex frequency domain $s$. 
Due to its linearity and its ability to convert derivatives into algebraic expressions in $s$, the resulting equation involves only algebraic terms, which are easier to manipulate. 
After simplification in the $s$-domain (see general shortcuts), the inverse Laplace transform is applied to obtain the solution in the time domain.

\begin{center}
\begin{tikzpicture}[
  box/.style={draw, rectangle, minimum width=3.6cm, minimum height=1cm, align=center},
  arrow/.style={->, thick}
]

\node[box] (de) {Differential\\Equation};
\node[box, below=of de] (alg) {Algebraic Equation\\in $s$};
\node[box, right=of alg] (sol) {Solution\\$f(t)$};

\draw[arrow] (de) -- node[right] {$\mathcal{L}$} (alg);
\draw[arrow] (alg) -- node[above] {$\mathcal{L}^{-1}$} (sol);

\end{tikzpicture}
\end{center}




\textbf{Derivatives in $s$-Domain}

Transform: $y(t) \to Y(s)$

First derivative: $\mathcal{L}\{y'(t)\} = sY(s) - y(0)$

Second derivative: $\mathcal{L}\{y''(t)\} = s^2Y(s) - sy(0) - y'(0)$

Third derivative: $\mathcal{L}\{y'''(t)\} = s^3Y(s) - s^2y(0) - sy'(0) - y''(0)$

\textbf{Multiplication by $t^n$:}
$\mathcal{L}\{t^n g(t)\} = (-1)^n \frac{d^n}{ds^n}F(s)$

\hrulefill


\textbf{Heaviside and Step Functions}

Not all functions encountered in engineering are continuous. 
Step functions are piecewise-defined functions that change their value or functional form at specified points along the $x$-axis. 
They are commonly used to model systems with sudden inputs or switching behaviors.

Turn off at $a$: $\begin{cases} g(t), & t<a \\ 0, & t\geq a \end{cases} = g(t) - g(t)\,u(t-a)$

Turn on at $a$: $\begin{cases} 0, & t<a \\ g(t), & t\geq a \end{cases} = g(t)\,u(t-a)$

Switch at $a$: $\begin{cases} g(t), & t<a \\ h(t), & t\geq a \end{cases} = g(t) + [h(t)-g(t)]\,u(t-a)$

\hrulefill

\textbf{Properties of Laplace functions}

Convolution: $(f*g)(t) = \displaystyle\int_0^t f(\tau)\,g(t-\tau)\,d\tau$

Laplace of convolution: $\mathcal{L}\{f*g\} = F(s) \cdot G(s)$

Integration property: $\mathcal{L}\left\{\displaystyle\int_0^t g(\tau)\,d\tau\right\} = \dfrac{1}{s}F(s)$

\textbf{First Shifting Theorem ($s$-shifting)}

If $\mathcal{L}\{f(t)\} = F(s)$, then: $\mathcal{L}\{e^{at}f(t)\} = F(s-a)$

Inverse: $\mathcal{L}^{-1}\{F(s-a)\} = e^{at}f(t)$

Common forms:
$\mathcal{L}\{e^{at}t^n\} = \frac{n!}{(s-a)^{n+1}}$, \quad
$\mathcal{L}\{e^{at}\sin kt\} = \frac{k}{(s-a)^2+k^2}$, \quad
$\mathcal{L}\{e^{at}\cos kt\} = \frac{s-a}{(s-a)^2+k^2}$


\textbf{Second Shifting Theorem}

Forward: $\mathcal{L}\{u(t-a)\,g(t-a)\} = e^{-as}F(s)$

Inverse: $\mathcal{L}^{-1}\{e^{-as}F(s)\} = u(t-a)\,g(t-a)$

\textbf{Rewriting $g(t)\,u(t-a)$:} Can't apply shift theorem directly! Rewrite $g(t)$ in terms of $(t-a)$:
\begin{align*}
t\,u(t-1) &= [(t-1)+1]\,u(t-1) = (t-1)u(t-1) + u(t-1)
\end{align*}
Now each term fits the shift theorem:
$\mathcal{L}\{(t-1)u(t-1)\} = e^{-s}/s^2$, \quad $\mathcal{L}\{u(t-1)\} = e^{-s}/s$

\hrulefill

\textbf{Periodic (Repeating) Functions}

For $g(t)$ with period $T$:
$\mathcal{L}\{g(t)\} = \frac{1}{1-e^{-sT}} \cdot \int_0^T e^{-st}g(t)\,dt$

If the function changes form, split the integral (multiplier stays):
$\mathcal{L}\{g(t)\} = \frac{1}{1-e^{-sT}} \cdot \left[\int_0^a e^{-st}g_1(t)\,dt + \int_a^{T} e^{-st}g_2(t)\,dt\right]$


\hrulefill


\textbf{Tips for Inversion}

\begin{enumerate}
\item \textbf{Completing the Square}

For $s^2 + bs + c$, rewrite as
\[
\left(s+\tfrac{b}{2}\right)^2 + \left(c-\tfrac{b^2}{4}\right).
\]
Example:
$s^2 + 6s + 34 = (s+3)^2 + 25$

then apply the $s$-shifting theorem with $a=-3$ and $k=5$.

\item \textbf{Partial Fraction Decomposition}

Decompose rational functions into simpler terms:
\[
\frac{1}{(s+a)^n}
= \frac{A_1}{s+a} + \frac{A_2}{(s+a)^2} + \cdots + \frac{A_n}{(s+a)^n}.
\]

\item \textbf{Standard Transform Matching}

Before decomposing, check whether the expression matches a known Laplace pair (exponentials, sines, cosines, polynomials). This often avoids unnecessary algebra.
\end{enumerate}


\hrulefill

\textbf{Solving Systems of ODEs}

\textbf{Method:}
(1) Take $\mathcal{L}$ of both equations
(2) Substitute ICs: $\mathcal{L}\{x'\} = sX - x(0)$, etc.
(3) Get algebraic system in $X(s), Y(s)$
(4) Solve for $X(s)$ and $Y(s)$ (elimination or Cramer's)
(5) Inverse transform to get $x(t), y(t)$

\textbf{Example setup:} For $\begin{cases} 2x' + y' - 2x = 1 \\ x' + y' - 3x - 3y = 2 \end{cases}$

After $\mathcal{L}$: $\begin{cases} (2s-2)X + sY = \frac{1}{s} + \text{ICs} \\ (s-3)X + (s-3)Y = \frac{2}{s} + \text{ICs} \end{cases}$

\hrulefill


\textbf{Matrix exponentials:} 
Laplace transforms can also be used for calculating exponentials of matrices as:

1. Compute $sI - A$ and $\det(sI - A)$

2. $(sI - A)^{-1} = \dfrac{\text{adj}(sI - A)}{\det(sI - A)}$

3. Partial fractions on each entry

4. $e^{At} = \mathcal{L}^{-1}\{(sI - A)^{-1}\}$
