\twocolumn[
\begin{center}
\textbf{\Large Partial Differential Equations}
\end{center}
\vspace{4pt}
]
%----------------------------------------------
\textbf{Partial Differential Equations (PDEs)} arise in connection with various physical and geometrical problems when the functions involved depend on two or more independent variables. The order of the highest derivative is called the order of the equation. In general, the solution space to any given PDE is very large, requiring us to define boundary conditions (e.g., $f(0,t)$) and initial conditions (e.g., $f(x,0)$) to achieve a useful solution.

{\small
\begin{tabular}{@{}ll@{}}
\toprule
Method & When Useful \\
\midrule
Separation of Variables & Simple geometries, homogenous BCs \\
Fourier Series/Transform & Periodic/infinite domains \\
Laplace Transform & Time-dependent, IVPs \\
Green's Functions & Inhomogeneous PDEs \\
Method of Characteristics & 1st-order \& hyperbolic \\
Finite Difference & Complex geom., any PDE \\
Finite Element & Irregular domains \\
\bottomrule
\end{tabular}
}


\textbf{Classification of 2nd-Order Linear PDE:} 

$Au_{xx} + Bu_{xy} + Cu_{yy} + Du_x + Eu_y + Fu = G$

\begin{tabular}{@{}lll@{}}
\toprule
Type & Condition & Example \\
\midrule
Hyperbolic & $B^2 - 4AC > 0$ & Wave eq. \\
Parabolic & $B^2 - 4AC = 0$ & Heat eq. \\
Elliptic & $B^2 - 4AC < 0$ & Laplace eq. \\
\bottomrule
\end{tabular}


\textbf{Separation of Variables}\\
\textbf{Steps:}
1. Assume $u(x,t) = X(x)T(t)$ and substitute into PDE.
2. Divide both sides to separate variables: $\frac{\text{terms in }t}{\text{coeff}} = \frac{\text{terms in }x}{\text{coeff}} = -\lambda$
3. Check three cases for spatial ODE $X'' + \lambda X = 0$:

\begin{tabular}{@{}lll@{}}
$\lambda < 0$ & $X = c_1 e^{\sqrt{-\lambda}x} + c_2 e^{-\sqrt{-\lambda}x}$ & (or $\sinh$/$\cosh$) \\
$\lambda = 0$ & $X = c_1 + c_2 x$ & \\
$\lambda > 0$ & $X = c_1\cos\sqrt{\lambda}x + c_2\sin\sqrt{\lambda}x$ & \\
\end{tabular}

4. Apply homogeneous BCs to find which $\lambda$ gives nontrivial solutions $\Rightarrow$ eigenvalues $\lambda_n$, eigenfunctions $X_n(x)$.
5. Solve temporal ODE for each $\lambda_n$ to get $T_n(t)$.
6. Form general solution: $u(x,t) = \sum_{n} c_n X_n(x)T_n(t)$
7. Apply IC to find coefficients $c_n$ using orthogonality.

\begin{tabular}{@{}ll@{}}
\toprule
BCs & $X_n(x)$, $\lambda_n$ \\
\midrule
$u(0,t)=u(L,t)=0$ & $\sin\frac{n\pi x}{L}$, $\lambda_n = \frac{n^2\pi^2}{L^2}$ \\[4pt]
$u_x(0,t)=u_x(L,t)=0$ & $\cos\frac{n\pi x}{L}$, $\lambda_n = \frac{n^2\pi^2}{L^2}$ \\[4pt]
$u(0,t)=u_x(L,t)=0$ & $\sin\frac{(2n-1)\pi x}{2L}$, $\lambda_n = \frac{(2n-1)^2\pi^2}{4L^2}$ \\
\bottomrule
\end{tabular}

\textbf{2D Problems} (e.g., $u_t = K(u_{xx}+u_{yy})$ on $[0,b]\times[0,c]$):

Assume $u = X(x)Y(y)T(t)$, separate $\Rightarrow$ eigenvalues in both $x$ and $y$.
\vspace{-0.5em}
\begin{multline}
u(x,y,t) = \sum_{n=1}^{\infty}\sum_{m=1}^{\infty} A_{nm}\sin\frac{n\pi x}{b}\sin\frac{m\pi y}{c} \\
\times e^{-K[(n\pi/b)^2+(m\pi/c)^2]t}
\end{multline}

\textbf{Finding $A_{nm}$:} At $t=0$: $u(x,y,0) = f(x,y)$. Multiply both sides by $\sin\frac{n\pi x}{b}\sin\frac{m\pi y}{c}$, integrate over domain, use orthogonality:
\[A_{nm} = \frac{4}{bc}\int_0^b\int_0^c f(x,y)\sin\frac{n\pi x}{b}\sin\frac{m\pi y}{c}\,dy\,dx\]

\textbf{1D Case with Constant IC:} If $u(x,0) = 1 = \sum_{n=1}^{\infty} B_n\sin(n\pi x)$

Multiply by $\sin(m\pi x)$, integrate $0$ to $1$: $\int_0^1 \sin(m\pi x)\,dx = \frac{1-(-1)^m}{m\pi} = \frac{1}{2}B_m$

$B_m = \frac{2(1-(-1)^m)}{m\pi} = \begin{cases} 0 & m \text{ even} \\ \frac{4}{m\pi} & m \text{ odd}\end{cases}$

Solution: $u = \sum_{n=1,3,5,\ldots} \frac{4}{n\pi}\sin(n\pi x)e^{-n^2\pi^2 kt}$ or $u = \sum_{k=1}^{\infty} \frac{4}{(2k-1)\pi}\sin((2k-1)\pi x)e^{-(2k-1)^2\pi^2 kt}$


\textbf{Tip:} $\int \frac{f'(x)}{f(x)}\,dx = \ln|f(x)| + C$

\hrulefill

\textbf{Systems of PDEs / Method of Characteristics}\\
Given: $\mathbf{u}_t + A\mathbf{u}_x = 0$. Diagonalize: $A = PDP^{-1}$. Substitute $\mathbf{v} = P^{-1}\mathbf{u}$:

$\mathbf{v}_t + D\mathbf{v}_x = 0$ $\Rightarrow$ decoupled equations $v_{i,t} + \lambda_i v_{i,x} = 0$

\textbf{Solve $v_t + cv_x = 0$:} Characteristics: $\frac{dx}{dt} = c \Rightarrow x - ct = \text{const}$

Solution: $v = F(x - ct)$. \quad If $v_t - cv_x = 0$: $v = F(x + ct)$

After solving, transform back: $\mathbf{u} = P\mathbf{v}$

\hrulefill

\textbf{Laplace Transform Method for PDEs}\\
$\mathcal{L}\{u_t\} = sU(x,s) - u(x,0)$

$\mathcal{L}\{u_{tt}\} = s^2U(x,s) - su(x,0) - u_t(x,0)$

$\mathcal{L}\{u_x\} = \frac{\partial U}{\partial x} = U_x(x,s)$

$\mathcal{L}\{u_{xx}\} = \frac{\partial^2 U}{\partial x^2} = U_{xx}(x,s)$

$\mathcal{L}\{u_{xt}\} = sU_x(x,s) - u_x(x,0)$

\textbf{Example: Heat Equation via Laplace Transform}

PDE: $u_t = ku_{xx}$, $u(x,0) = f(x)$, $u(0,t) = 0$, $u(L,t) = 0$

Apply $\mathcal{L}$ in $t$: $sU - f(x) = kU_{xx}$

Rearrange: $U_{xx} - \frac{s}{k}U = -\frac{f(x)}{k}$ (ODE in $x$)

Solve ODE, apply BCs, then invert.

\hrulefill

\textbf{Duhamel's Principle}
For nonhomogeneous PDE with zero ICs: $u_t = ku_{xx} + F(x,t)$ or $u_{tt} = c^2u_{xx} + F(x,t)$.
The idea is to treat the source term $F(x,\tau)$ as an initial condition at time $\tau$ and superpose solutions over all past times.

\textbf{Steps:}
1. Solve the corresponding homogeneous problem with IC $= F(x,\tau)$ at $t = \tau$.
2. Call this solution $v(x,t;\tau)$ (parameterized by $\tau$).
3. Final solution: $u(x,t) = \int_0^t v(x,t;\tau)\,d\tau$

\hrulefill

\textbf{Sturm-Liouville Problems}\\
Form: $\frac{d}{dx}\left[p(x)\frac{dy}{dx}\right] + [q(x) + \lambda r(x)]y = 0$

Properties: Eigenvalues $\lambda_n$ are real; eigenfunctions $y_n$ orthogonal with weight $r(x)$:
\[\int_a^b r(x)y_n(x)y_m(x)\,dx = 0 \quad (n \neq m)\]

Eigenfunction expansion: $f(x) = \sum_{n=1}^{\infty} c_n y_n(x)$

$c_n = \frac{\int_a^b r(x)f(x)y_n(x)\,dx}{\int_a^b r(x)y_n^2(x)\,dx}$